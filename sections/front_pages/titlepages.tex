\pdfbookmark[0]{English title page}{label:titlepage_en}
\aautitlepage{%
  \englishprojectinfo{
    Balancing rights to privacy with protection of society %title
  }{
     IT Security Governance %theme%
  }{%
    Spring 2025  %project period
  }{%
    Group 803 % project group
  }{%
    %list of group members
     
    Kamrul Islam\\
    kislam24@student.aau.dk\\
    Khagendra Thakurathi\\
    kthaku24@student.aau.dk\\
    Mohamod Ansari\\
    mansar24@student.aau.dk\\
    Newton Sharma\\
    nsha24@student.aau.dk\\
    Rajeev Thapa\\ 
    rthap24@student.aau.dk\\
 }{%
    %list of supervisors
    Qiongxiu Li\\
    qili@es.aau.dk 
  }{%
    1 % number of printed copies
  }{%
    \today % date of completion
  }%
  
}{%department and address  
}{% the abstract
\footnotesize
Nowadays governments, organizations, and citizens are concerned about how individual's privacy and security can be balanced. So, in this report our main aim is to examine the complex relationship between protection of privacy and cybersecurity requirements within the European Union's regulatory framework, the General Data Protection Regulation (GDPR) and the Network and Information Systems Directive 2 (NIS2). This report explores how technical security solutions, legal frameworks, and governance mechanisms can be used to protect both individual privacy and security through a systematic literature review process, legal analysis, and examination of real-world case studies. The study identifies key challenges in implementing security measures that often require extensive  collection of data, which may conflict with fundamental  principles of privacy  data minimization, purpose limitation, and informed consent. \\
\noindent The report analyzes the evolving threats to privacy, such as ransomware, phishing,  vulnerabilities in IoT, supply-chain breaches, and emerging AI-driven privacy threats. It discusses commonly used Privacy Enhancing Technologies (PETs)including differential privacy, homomorphic encryption, and secure multi-party computation, and inherent trade offs based on standards and regulatory guidance. Four case studies show practical challenges and potential solutions in balancing these competing interests.The analysis shows that privacy and security are not contradictory but are complementary objectives that require combination of technical and legal approaches. Furthermore, this report suggests policymakers, implementers, and organizations how this balance can be obtained through principles of Privacy by Design, transparent governance and effective communication between stakeholders. This research contributes to the ongoing debate on how security and privacy can go hand on hand, offering a platform for developing trustworthy and strong digital ecosystem that protect both individual freedoms and security.
}
