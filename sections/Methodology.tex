\chapter{Methodology} \label{cha:Methdology}
\subsection* {Literature Search Strategy}

\noindent To collect relevant academic and industry sources for this project, we followed a structured literature search process using multiple academic databases. The primary platforms used were Google Scholar, Scopus, and the IEEE Digital Library, as these offered broad coverage of peer-reviewed articles related to cybersecurity, privacy, digital governance, and information systems. The initial searches were performed using simple keyword queries such as "privacy and security balance", "GDPR cybersecurity", "NIS2 directive", "privacy-enhancing technologies", and "cyber threats affecting privacy". After narrowing the focus of the project, we applied advanced search operators (such as AND and OR) to combine keywords for example, "privacy AND cybersecurity governance" or "ransomware OR data breach AND healthcare". This helped filter irrelevant results and ensured that the selected papers aligned closely with the theme of balancing privacy rights with the protection of society. \\

\noindent In Scopus, we used built-in filtering tools such as year range, subject area, document type, and keyword mapping. Since the aim was to work with recent and relevant research, we set the publication window to 2018–2025, ensuring that the material reflected the latest trends, regulations, and technologies in cybersecurity and privacy. However, in specific cases where no recent studies existed such as the 2017 NHS ransomware attack, which remains a major real-world reference point and we included slightly older sources when they provided essential context for understanding how privacy and security conflicts manifest in practice. In the IEEE Digital Library, additional filters were applied to include only peer-reviewed journals, excluding magazines, letters, dissertations, and conference summaries. This approach ensured the technical accuracy and academic credibility of the selected materials. 

\subsection* {Analytical Framework}

\noindent After the selection of relevant sources, a thematic analysis approach was used to interpret and synthesize the literature. Each paper was reviewed and categorized according to key themes central to the research question, including regulatory frameworks (such as GDPR and NIS2), the evolving landscape of cybersecurity threats, the role of privacy-enhancing technologies, organizational governance and compliance practices, and the ethical considerations involved in balancing privacy with societal security. Grouping the literature into these thematic areas made it possible to compare the perspectives in relation to the technical, legal, and policy fields in a structured manner and identify the common patterns, new challenges, and gaps in the current research. This analytical framework offered a clear-cut basis in which various insights have been incorporated and a developed outlook of the trade-offs between the privacy rights and security demands has been formulated. 

\subsection* {Biases and Limitations}
\noindent Several limitations were recognized throughout the research process. To begin with, the review was based only on the scholarly sources published in English, which could cause the language bias and omit the studies that might be related and conducted in other parts of the world. Moreover, cybersecurity and privacy are dynamic areas, and thus some of the insights presented are likely to become obsolete soon as new technologies and threats are introduced, as well as new interpretations of regulations are provided. There is a lack of empirical research on newly introduced frameworks, especially the NIS2 Directive, which makes real-life evaluations less available. All that can limit the broader applicability of the findings but the recognition of such factors offers transparency and context to the conclusions made. 