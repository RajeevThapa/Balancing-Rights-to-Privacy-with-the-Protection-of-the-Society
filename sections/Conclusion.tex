\chapter{Conclusion and further research} \label{cha:Conclusion and further research}
\section{Conclusion}

\noindent The project has explored the increasingly complex challenge of achieving an essential balance between robust collective cybersecurity and the protection of individual privacy in a digitalized world. The central conclusion emerging from the analysis of regulatory frameworks, technical mechanisms, and the evolving threat landscape is that this equilibrium cannot be reached through isolated or one-dimensional strategies. Rather, it involves persistent incorporation of technical resiliency, moral codes, and adaptive governance. Although security seeks to protect the systems and ensure confidentiality and integrity of data, privacy is focused on giving individuals the control and freedom of their personal information. The ongoing contradiction between these goals points to the necessity to shift towards more proactive system design where security and privacy do not oppose each other but support each other. \\

\noindent The practical implications of this challenge are clearly illustrated by the case studies. The British Airways data breach identified a very important failure of security practices in the private-sector in which lack of proper control over third-party supply chains resulted in a significant breach of customer data. This accident re-established the principle of accountability of the organization in the regulation and proved that the efficiency of operations can never be used in place of basic technical protective measures. Conversely, the use of COVID-19 contact-tracing apps is an example of how collective security requirements and individual rights to privacy can be mutually satisfactory. These systems achieved both the necessary public health functionality and preserved user trust by providing important public health functionality through decentralized architectures and by using privacy-enhancing technologies in accordance with privacy-by-design requirements. Their lead is a demonstration that the collective security is possible without the loss of data protection as long as the transparency and user agency are in the middle. \\

\noindent However, the analysis of state surveillance and suggested regulatory intervention shows long-term systemic risks.  The expanding use of facial recognition technology in law enforcement raises concerns about mission creep and algorithmic bias, where efforts to bolster public safety risk encroaching upon civil liberties and violating principles of non-discrimination. These problems highlight the importance of strict regulatory control and extensive moral assessment. What is more importantly problematic is that even the EU Chat Control proposal reveals a more profound tension of structure: the fact that it is based on client-side scanning, despite its protective nature, puts the confidentiality, which is inherent to the end-to-end encryption, at risk. This demonstrates how mandatory, widespread monitoring can erode proportionality, introduce new security vulnerabilities, and diminish the trust required for a functional digital society.\\

\noindent Finally, the project confirms that the development of a trustful and sustainable digital ecosystem is based on the strict application of integrated solutions. These are integrating Privacy by Design into the design, enhancing the use of Privacy-Enhancing Technologies, implementing Zero Trust designs, and promoting open communication in order to maintain the trust of the public. The case studies as a whole demonstrate that privacy and security lapses are as many as governance and ethics. The way forward should then focus on hybrid technical solutions that will balance utility with protection coupled with globally coordinated regulations, which are proactive, aligned, and dedicated to making sure that the security potentials are never at the cost of the core individual rights.

\section{Further Research}
\noindent Further research is essential to address the structural gaps in balancing privacy and security, particularly in the development of integrated technological solutions. A significant research gap still exists in developing comprehensive protection technologies that can provide privacy, security and reliability in various applications as there is not a single mechanism that can offer all three features. The development of complete and hybrid protection schemes (including K-anonymity and blockchain or a combination of Blind Approach techniques and blockchain) should be given more priority in future studies to strengthen both privacy and security. Further development of privacy-enhancing technologies is also of great importance, such as improving the technique of differential privacy and the use of federated learning to deal with more complex re-identification attacks. \\

\noindent In addition to technological innovation, the future research should focus on adaptive governance and the measurement of organizational resilience in changing regulatory environments. The correlation between cybersecurity resilience and the level of digitalization is a topic that lacks adequate research on the issue at hand. The policymakers also need evidence-based information to develop adaptive regulatory frameworks that encourage innovation and at the same time protect the rights of users. In addition, the absence of empirical research about newly created legal tools, specifically the NIS2 Directive, restricts practical measurement of it and limits the extent to which existing results may be applicable. The question of AI governance is another priority that requires research providing the empirical recommendations on how to reduce the risk of AI leakage, the vulnerabilities of human-AI interactions, and the obscurity of the supply chain in the context of AI.