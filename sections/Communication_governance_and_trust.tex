\chapter{Communication, Governance and Trust} \label{cha:Communication, governance and trust}

% Please, Include the following:
% \\\\
% risk communication, transparency, accountability and governance models
\subsection* {Risk Communication}
Communication risk is understood as the interactive process of exchanging information about a risk including its nature, meaning, consequences, likelihood, and response options in order to enable individuals to make informed judgments. This activity has goals such as promotion or transformation of knowledge and attitudes, alteration of risk-relevant behavior, and cooperative resolution of conflicts and decision-making \cite{nurse2011trustworthy}. The particular problem of cybersecurity risk communication is the difficulty in identifying the optimal way of presenting security-risk information to people to make them understand it and exercise good judgment in security-related matters. Nonetheless, risk communication may be regarded as a complicated issue because of the various factors that influence the risk message or message communicator or source, and the recipient. Risk details are often complex, and ambiguous or imprecise information can only make one anxious not make him/her become more aware of the risk. The message structure is also vital; whereas numerical formats are accurate, they are also dependent on some degree of mathematical skills (numeracy) that even highly educated people do not possess.\\

\noindent Nurse et. al state that the effectiveness of communication is highly influenced by the characteristics of the message receiver, such as their culture, beliefs, knowledge, familiarity with risk, sense of control, perceived impact, and emotional state (e.g., fear, anger, or anxiety). Therefore, to improve the communication of cybersecurity risks, system designers should aim to make information easier to process by reducing cognitive effort and simplifying the message \cite{nurse2011trustworthy}. Most importantly, messages must be delivered in a timely manner, preferably near the risk scenario or attack, in a common and predictable format, and it should be clear what the messages are all about to enable it to be verified or linked to a source.\\

\noindent Cyberattacks are one of the factors that shape people's perception of regulatory processes and contribute to the increased popularity of strict cybersecurity regulations. This is mediated by the perception of threat by the public. The perceptions of cyber threat are elevated to a higher level than the perceptions of nonlethal or economic cyberattacks caused by exposure to lethal cyberattacks \cite{snider2021cyberattacks}. The public interest comes into the limelight in the wake of such incidents and it is followed by demands of government intervention, and people have been ready to lay down civil liberties and privacy in favor of increased security.

\subsection* {Governance and Governance Models}
A legal and regulatory compliance defines governance in the area of the digital security and personal data. The EU regulatory setting, an example of which is the GDPR, eIDAS, and NIS2, obliges data processors to administer and execute the rights of data subjects, in relation to the way their data is disclosed and utilized. The latest policy initiatives defining the data governance are the Digital Services Act (DSA), Digital Markets Act (DMA), and the Data Governance Act (DGA) \cite{kyriakoulis2025consentis}.

Public exposure to cyberattacks dictates support for specific types of governmental governance models or policies, which reflect the delicate tradeoff between security and privacy. According to Snider et al., three primary regulatory policy dimensions of cybersecurity can be distinguished \cite{snider2021cyberattacks}:
\begin{enumerate}
    \item Cybersecurity Prevention Policy (CPP): This is based on the state imposing minimum standards of cybersecurity on commercial companies to avoid harm. This research concluded that the effect of support on CPP was only mediated by threat perception but not by the type of exposure.
    \item Cybersecurity Oversight Policy (COP): This model is the direct intervention of the state to provide cyber protection to people and companies. Contact to nonlethal cyberattacks also foretell the support of Oversight Policies.
    \item Cybersecurity Alert Policy (CAP): This type of model presupposes that the state ensures users are notified in case a hack or a cyberattack is found. Individuals that experience cyberattack can be influenced by deaths in an attempt to support Alert Policies at elevated levels.
\end{enumerate}

These EU regulations and strategic digital data initiatives are supported using the CONSENTIS framework that is aimed at providing legal compliance and security by having a mechanism to assess the process of legal compliance on a continuous basis. Its Legal and Deviations Assessment module measures the data usage and privacy policy against Smart Contracts to certify that it meets the standards, such as GDPR, the Data Governance Act, and the Data Act, and also the compliance with ethical standards, such as ALTAI \cite{kyriakoulis2025consentis}.

\subsection* {Trust}
Trust is a fundamental element necessary for effective risk communication and for the successful adoption of new data management systems \cite{nurse2011trustworthy}. People require more control, visibility, and safety to their personal information and systems offering such a feature are necessarily able to build trust.

Regarding risk communication, it is necessary to know what factors affect perceived trustworthiness. The factors of information trustworthiness are classified according to the information source, the piece of information, or the end-user. The factors related to the source that are important are the source identity, his/her reputation, credentials, and good intentions. Nurse et. al pointed that accuracy, believability, consistency, relevance, specificity, timeliness, and presentation/format are the key factors of information \cite{nurse2011trustworthy}. Expertise, motivation, and familiarity are among the end-user factors. In case one lacks trust in the origin of a message about a security risk, there is a high probability that one would not perceive the risk appropriately and this could result in either an overestimation or underestimation of the risk. An accurate, specific, presented and familiar security-risk message stands a better chance of being credited and acted on.

When it comes to data governance, CONSENTIS enables trust and compliance using Distributed Ledger Technologies (DLT) to build a decentralized infrastructure that guarantees integrity and authenticity of the information stored in it, as well as associated actions. Through the automated consent management, the framework seeks to support scalable and reliable data sharing throughout the EU single market \cite{kyriakoulis2025consentis}.

\subsection* {Transparency and Accountability}
Transparency is a core requirement of data protection regulations like GDPR, which grants data subjects the Right of Access (Art. 15) to their personal information \cite{kyriakoulis2025consentis}. Individuals specifically demand better control, transparency, and security regarding their personal data. CONSENTIS mitigates this by proposing transparent, secure and privacy enhancing mechanisms of consent management. It seeks to provide users with complete control of their information, which is informed consent, selective disclosure of data, and real-time transparency.

The goal of CONSENTIS is to facilitate data usage in a way that guarantees that stakeholders have the ability to locate, request, access, and use personal data in a transparent and lawful way \cite{kyriakoulis2025consentis}. Emerging solutions on how to use SSIs are heading toward the direction of finding more transparent solutions to consent and management. Additionally, the Privacy and Data Usage Control module will provide transparent access control, policy-driven access control so that the user can track the access to his or her data and the right to access it.

Transparency is directly connected to accountability that is one of the concerns when handling personal data. Conventional centralized consent management solutions have elicited certain concerns as per the absence of accountability. CONSENTIS is created as an active and open consent management system and provides accountability and compliance. The framework achieves tamper-proof traceability, which supports accountability, by recording consent transactions and user identities on a blockchain via the Consent Policy Manager \cite{kyriakoulis2025consentis}.