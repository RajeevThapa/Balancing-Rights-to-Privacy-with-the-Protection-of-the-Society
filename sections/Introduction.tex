\chapter{Introduction} \label{cha:introduction}
 Today's digital worlds main problem is how to protect people from cybersecurity threats without compromising their fundamental right to privacy\cite{dalal2020cybersecurity}.Nowadays, people are too dependent on digital technology to perform their day to day activities, which has resulted in  cybercriminals, state-sponsored actors, and malicious insiders exploiting  vulnerabilities which  threat  individuals, organization and essential  sectors such as healthcare, energy, financial institutions, government services, and many other services. 
\\

\noindent In order to defend against these threats, governments and organizations are implementing security measures such as network monitoring, data collection, threat detection system, and  technologies to monitor the activities of cyber criminals. However, these solutions are  creating  a problem: the same tool designed to protect individual can harm the privacy rights they aim to protect\cite{cole2019interplay}. When security systems monitor network traffic, analyze behaviour of user, or collect large amounts of data to detect threats, they can automatically access personal information and track the activities of people.\\


\noindent Due to this it creates a tension between two essential components, security and privacy. On the one hand, governments and organizations has right to prevent cyberattacks that can disrupt critical services, steal sensitive information, and harm citizens. On the other hand, individuals have fundamental rights to privacy, data protection, and freedom from unwanted monitoring that are expressed in regulations such as GDPR and are supported by directives like NIS2, whose main aim is to strengthen security across critical sectors. \\

\noindent Recent incidents show the challenge of data breaches that have exposed millions of personal records, ransomware attacks has resulted in public services to malfunction, and sophisticated cyber-espionage operations threaten national security. In response to these activities, governments and organizations have implemented  comprehensive security measures, but these measures also raise questions like How much monitoring is acceptable? Who has access to the collected data? For how long should the data be retained? and how can we ensure that security measures help prevent the rights of an individual? \\

\noindent The main purpose of this paper is to examine the relationship between security and privacy in a digitalized society and to analyze how technical security solutions and regulatory frameworks can balance the protection of personal data with security needs. The paper discusses the security threats that challenge privacy and explains their technical operation and impact. It analyzes how GDPR and NIS2 manage the balance between privacy protection and security obligations. The paper explains PETS intended function and inherent trade offs based on standards and regulatory guidance. It uses four case studies that illustrate real-world situations where security measures, regulatory obligations, and privacy issues interact. The study also examines how communication, transparency, and trust influence how users understand and accept cyber security measures.
\\

\noindent To better understand this balance, it is important that we first understand what security and privacy are in the digital context. Both concepts are interrelated, but their  purpose is different: security focuses on protecting systems and data, while privacy  is concerned with the rights and freedoms of individuals.

\section{Overview of Security and Privacy Concepts }

Security and privacy are two essential and distinct ideas. They are often mentioned together, each dealing with different set of issues related to protecting information and systems. Security focuses on protecting systems and data from external threats, while privacy focuses on protecting individuals rights to  their personal information such as their name, image, and other personal records. Understanding the relationship between these two concepts is crucial for designing durable technologies and increasing trust in digital ecosystem\cite{achuthan2024advancing}. 
\subsection{Security}
Security means protection of system, data and networks from unauthorized access, damage, or disruption.This protection is shaped around three core principles known as the CIA triad: confidentiality, integrity, and availability\cite{samonas2014cia}. 
Confidentiality ensures that sensitive information remains accessible only to authorized individuals or systems, preventing it from unauthorized disclosure,  integrity makes sure that the data is accurate by preventing unauthorized modifications and ensure that the information remains reliable and availability guaranties that systems and data remain accessible to authorized users when needed\cite{inbook}.
\\
\noindent To achieve these objectives, organizations implement various security measures, such as firewalls, encryption techniques, intrusion detection systems, IAM, and multifactor authentication. Moreover, security measures are continuously improving on a daily basis to address  emerging threats such as phishing, malware, ransomware, and APTs, which require constant monitoring and adaptation.


\subsection{Privacy} 

 The main focus of security is to protect systems and data, whereas privacy focuses on protecting people. Privacy refers to the fundamental right of individuals to control the way their personal information  is collected, stored, shared, and used. 
\\
Privacy is shaped by key principles like data minimization, consent, and transparency. Data minimization allows the organization to collect only the data necessary for a specific purpose, avoiding excessive collection which increases the chances of risk in future. Consent make sure that the individual has granted  permission  to perform actions on their data and transparency ensures that the operations are clear.
\\
This distinction between security and privacy is important because  a system can be highly secure but it can still violate privacy if it collects or uses personal data without taking a proper consent from an individual whose data is being collected or processed.Security is technical in nature, while privacy is linked to cultural norms, legal frameworks, and ethical considerations. Regulations like GDPR reflect this complexity by establishing comprehensive rules for data handling that balance needs of an organization with individual rights and freedom\cite{VANDERSCHYFF2024104065}. 
\section{Structure of the Paper}

This report is structured into 11 chapters.
\\
Chapter 1 introduces the research problem and core concepts of security and privacy.In Section 1.1, security (Section 1.1.1) and privacy (Section 1.1.2) are defined to clarify their distinct and interconnected roles in digital systems.
\\
Chapter 2 provides a comprehensive review of literature. Section 2.1 introduces cybersecurity and privacy challenges in a digitalized society, while Section 2.2 examines different types of cybersecurity threat that challenge privacy and data protection. In section 2.3  legislative and regulatory frameworks are analyzed, focusing on GDPR and NIS2, followed by Section 2.4, in which  technical and ethical solutions are discussed such as Privacy-Enhancing Technologies (PETs). In section 2.5  the role of communication, trust, and public perception are discussed, and Section 2.6 combine the findings into a balanced framework for privacy and security, including a comparative analysis in Section 2.6.1 and in section 2.7 the key insights from the literature are summarized.
\\
Chapter 3 explains the methodology, describing how the analysis is carried out and how the sources are evaluated.
\\
Based on the reviewed literature, Chapter 4 analyzes the threat landscape and examines the major cyber threat and shows how they affect privacy.
\\
In Chapter 5 legal and regulatory analysis is performed.Section 5.1 examines the GDPR, including its fundamental concepts in subsection 5.1.1, while Section 5.2 focuses on the NIS2 Directive and its obligations in subsection 5.2.1. The interaction between GDPR and NIS2 is discussed in Section 5.3, which is then followed by a real-world example in Section 5.4.
\\
Chapter 6 evaluates technical solutions and standards.Section 6.1 introduces the principle of privacy by design, with subsections 6.1.1 and 6.1.2 addressing proactive measures and default privacy settings. Section 6.2 discusses Privacy-Enhancing Technologies, their applications (subsection 6.2.1) and trade-offs (subsection 6.2.2). Relevant standards and guidance are  then covered in Sections 6.3 and 6.4.
\\
Following the technical discussion, In Chapter 7 analysis of communication, governance, and trust is performed which is necessary for effective implementation of privacy and security measures.
\\
Chapter 8 presents four case studies. Sections 8.1–8.4 analyze real-world examples, which include  British Airways data breach, COVID-19 contact-tracing apps, facial recognition in law enforcement, and the EU Chat Control proposal.
\\
Chapter 9 provides recommendations, including policy and regulatory recommendations (Section 9.1), a technical implementation roadmap (Section 9.2), and communication and trust-building guidelines (Section 9.3).
\\ 
Chapter 10 concludes the report and outlines directions for future research in Sections 10.1 and 10.2, respectively.
\\
Finally, Chapter 11 explains how we have used generative AI tools in the preparation of this report.


